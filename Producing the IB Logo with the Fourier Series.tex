\documentclass[letterpaper, 12pt]{article}
\usepackage{graphicx}
\graphicspath{{Figures/}{./}}
\usepackage{apacite}
\usepackage{amsmath}
\usepackage{amssymb}
\usepackage{amsthm}
\usepackage{indentfirst}
\usepackage[justification=centering]{caption}
\usepackage{float}

\title{MATHEMATICS ANALYSIS AND APPROACHES HL
\\
Producing the IB Logo with the Fourier Series}
\author{Candidate Code:}
\date{}

\begin{document}
\maketitle
\newpage

\section{Rationale}

I have shown interest in visual arts done through the means of software,
with particular experience in 3D modelling and animation in Blender
and Cinema 4D.

I never was experienced with drawing, therefore producing digital art
on a 2D plane using artistic skill was not of interest to me. However,
something that I came across online was the use of the Fourier Series
in order to produce vector art, which instantly intrigued me.

While vector art files such as those with the file extension ".svg" relate
to mathematics in the sense that it contains multiple graphed mathematical
relationships in order to produce an image, the method of using the Fourier
Series to produce similar art is more mathematically intriguing, as it
proves use just one expression to produce the same result done by the numerous mathematical
relationships.


\section{Aim}

\section{Plan of Action}

% TODO: Use this citation at some point appropriately
% so Bibtex doesn't break
\cite{Sanderson}

\bibliographystyle{apacite}
\bibliography{References}

\end{document}